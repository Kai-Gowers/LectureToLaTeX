\documentclass[12pt]{article}
\usepackage{amsmath, amssymb}

\usepackage{amsmath}
\usepackage{amssymb}
\begin{document}

\section*{Quotient Group and Normal Subgroup}

The problem involves proving a statement about group homomorphisms and normal subgroups. Let \(\pi: G \to G/H\) be the natural projection where \(H\) is a normal subgroup of \(G\). We want to examine the implications of \(K/H \trianglelefteq G/H\) and how it affects the subgroup \(K\) of \(G\).

\subsection*{Forward Implication \((\Rightarrow)\)}

Assume \(\pi(K) \trianglelefteq G/H\). Let us demonstrate that this implies \(K \trianglelefteq G\).

Given: 
\[
K/H \trianglelefteq G/H
\]
It follows that for any coset \(gH \in G/H\), with \(\pi(g) \in G/H\), we have:
\[
\pi(g) \pi(K) \pi(g^{-1}) = \pi(K')
\]

\paragraph{Explanation:} 

Since \(\pi\) is a homomorphism:
\[
\pi(gkg^{-1}) = \pi(k')
\]
This implies:
\[
gkg^{-1} \in K
\]
Thus, \(GKH \subseteq K\), and so \(K \trianglelefteq G\).

\subsection*{Reverse Implication \((\Leftarrow)\)}

Now consider the reverse, assuming \(K \trianglelefteq G\), we need to show that \(K/H \trianglelefteq G/H\).

Assume \(K \trianglelefteq G\), therefore:
\[
gKg^{-1} = K \quad \text{for all} \quad g \in G
\]

This implies:
\[
\pi(g)\pi(K)\pi(g^{-1}) = \pi(K)
\]

\paragraph{Explanation:}

Since \(K\) is normal in \(G\), any conjugate \(gkg^{-1}\) remains in \(K\). This property transfers through the homomorphism \(\pi\) implying:
\[
K/H \trianglelefteq G/H
\]
Thus, proving \(\pi(K) = K/H \trianglelefteq G/H\).

\end{document}