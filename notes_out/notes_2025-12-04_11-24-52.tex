\documentclass[12pt]{article}
\usepackage{amsmath, amssymb}

\usepackage{amsmath}
\usepackage{amssymb}
\begin{document}

\section{Group Homomorphisms and Quotients}

In this discussion, we explore the relationship between subgroup homomorphisms and quotient groups.

\subsection{Statement and Setup}

Let \(\pi : G \to G/H\) be the canonical projection from a group \(G\) to the quotient group \(G/H\). We examine the condition for a subgroup \(K \leq G\) such that \(\pi(K) = K/H\). We assume \(K/H \trianglelefteq G/H\), that is, \(K/H\) is a normal subgroup of \(G/H\).

\subsection{Implication: \((\Rightarrow)\)}

Assume \(\pi(K) = K/H\). We need to show \(K \trianglelefteq G\).

\[
\pi(g)\pi(K)\pi(g^{-1}) = \pi(K)
\]

This implies:

\[
gKg^{-1} = K
\]

Hence, for \(k' \in K\), there exists \(h \in H\) such that:

\[
gkg^{-1} = k'h \quad \text{for some} \quad k' \in K, \, h \in H
\]

Since \(H \leq \ker \pi\), we have:

\[
h \in \ker \pi \quad \Rightarrow \quad H \leq K
\]

Thus, \(K \trianglelefteq G\).

\subsection{Converse: \((\Leftarrow)\)}

Assume \(K \trianglelefteq G\), i.e., \(gKg^{-1} = K\) for all \(g \in G\).

\[
\implies \pi(g)\pi(K)\pi(g^{-1}) = \pi(gKg^{-1}) = \pi(K)
\]

This implies:

\[
\pi(K) \leq \pi(\ker \pi) \quad \Rightarrow \quad K \leq \ker \pi
\]

Thus, \(\pi(K) = K/H \implies K/H \trianglelefteq G/H\).

\section{Conclusion}

We have shown that \(K \trianglelefteq G\) if and only if \(\pi(K) = K/H \trianglelefteq G/H\). This demonstrates a fundamental correspondence between subgroups of a group and its quotient.

\end{document}