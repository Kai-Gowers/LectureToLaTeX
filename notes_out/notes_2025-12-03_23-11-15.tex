\documentclass[12pt]{article}
\usepackage{amsmath, amssymb}

\usepackage{amsmath}
\usepackage{amssymb}
\begin{document}

\section*{Vector Transformation and Cross Product}

We aim to demonstrate the validity of the following equation involving vector transformations and cross products:

\[
\mathcal{S}(\mathbf{v}) \times \mathcal{S}(\mathbf{w}) = \mathbf{\hat{v}} \times \mathbf{\hat{w}}
\]

where \(\mathcal{S}(\mathbf{v})\) and \(\mathcal{S}(\mathbf{w})\) represent the transformed vectors with some corrections involving unit vectors and scalar factors.

\subsection*{Definitions and Assumptions}

First, we define the vector transformations:

\[
\mathcal{S}(\mathbf{v}) = \mathbf{\hat{v}} + (\mathbf{\hat{v}} \cdot \mathbf{\lambda}) \mathbf{s}
\]
\[
\mathcal{S}(\mathbf{w}) = \mathbf{\hat{w}} + (\mathbf{\hat{w}} \cdot \mathbf{\lambda}) \mathbf{s}
\]

Here, \(\mathbf{\hat{v}}\) and \(\mathbf{\hat{w}}\) are unit vectors, \(\mathbf{\lambda}\) is a vector, and \(\mathbf{s}\) is a scalar factor.

\subsection*{Cross Product Expansion}

To proceed, we evaluate the cross product \(\mathcal{S}(\mathbf{v}) \times \mathcal{S}(\mathbf{w})\).

\[
\begin{aligned}
\mathcal{S}(\mathbf{v}) \times \mathcal{S}(\mathbf{w}) &= \left[ \mathbf{\hat{v}} + (\mathbf{\hat{v}} \cdot \mathbf{\lambda}) \mathbf{s} \right] \times \left[ \mathbf{\hat{w}} + (\mathbf{\hat{w}} \cdot \mathbf{\lambda}) \mathbf{s} \right] \\
&= \mathbf{\hat{v}} \times \mathbf{\hat{w}} + \mathbf{\hat{v}} \times \left[(\mathbf{\hat{w}} \cdot \mathbf{\lambda}) \mathbf{s}\right] + \left[(\mathbf{\hat{v}} \cdot \mathbf{\lambda}) \mathbf{s}\right] \times \mathbf{\hat{w}} \\
&\quad + \left[(\mathbf{\hat{v}} \cdot \mathbf{\lambda}) \mathbf{s}\right] \times \left[(\mathbf{\hat{w}} \cdot \mathbf{\lambda}) \mathbf{s}\right]
\end{aligned}
\]

\subsection*{Simplification}

Upon simplification, it is shown that:

\[
\left[(\mathbf{\hat{v}} \cdot \mathbf{\lambda}) \mathbf{s}\right] \times \mathbf{\hat{w}} + \mathbf{\hat{v}} \times \left[(\mathbf{\hat{w}} \cdot \mathbf{\lambda}) \mathbf{s}\right] = \mathbf{0}
\]

Thus, we have:

\[
\mathcal{S}(\mathbf{v}) \times \mathcal{S}(\mathbf{w}) = \mathbf{\hat{v}} \times \mathbf{\hat{w}}
\]

\section*{Geometric Interpretation}

Let us consider the geometric interpretation by examining a parallelogram formed by \(\mathbf{\hat{v}}\) and \(\left(\mathbf{\hat{w}} \cdot \mathbf{\lambda} \right) \mathbf{s}\):

\[
\text{Area of } \Pi = h \times b
\]

which simplifies using the determinant properties:

\[
\begin{aligned}
\left|\mathbf{\hat{v}} \to \mathbf{s} + (\mathbf{\hat{w}} \to \mathbf{s})\right| &= \left|\mathbf{\hat{v}} \right| \cdot \left| \mathbf{\hat{w}} \circ \mathbf{s}\right| \\
&= \left|\mathcal{S}(\mathbf{v}) \times \mathcal{S}(\mathbf{w}) \right|
\end{aligned}
\]

This reinforces our algebraic demonstration with a visual confirmation.

\end{document}