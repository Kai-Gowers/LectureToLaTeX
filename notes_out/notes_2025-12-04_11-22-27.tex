\documentclass[12pt]{article}
\usepackage{amsmath, amssymb}

\usepackage{amsmath}
\usepackage{amssymb}
\begin{document}

\section*{Proof of a Group Theoretic Property}

We are given a group \( G \) and a normal subgroup \( H \) of \( G \). We are exploring a condition involving subgroups \( K \) of \( G \) and the quotient group \( G/H \). The goal is to prove the equivalence:
\[
\pi(K) = K/H \quad \text{if and only if} \quad K \trianglelefteq G.
\]

Here, \(\pi: G \to G/H\) is the canonical projection.

\subsection*{Forward Direction}

\((\Rightarrow)\) Assume that \(\pi(K) = K/H\). We want to show that \( K \trianglelefteq G \).

Start by letting \( \pi: G \to G/H \) be the natural projection. Suppose that \( \pi(K) = K/H \), which means that every coset \( gH \) such that \( g \in K \) satisfies the equality condition under projection. 

\[
\pi(g)\pi(K)\pi(g^{-1}) = \pi(K).
\]

This implies:

\[
gKg^{-1} \subseteq K.
\]

Hence, \( gk'g^{-1} = k'h \) for some \( h \in H \), and therefore \( K \) is normal in \( G \).

\subsection*{Backward Direction}

\((\Leftarrow)\) Assume that \( K \trianglelefteq G \). We need to prove that \(\pi(K) = K/H\).

Since \( K \trianglelefteq G \), for any \( g \in G \), \( gKg^{-1} = K \).

Therefore, if \( k \in K \), \( gkg^{-1} \in K \). It follows that

\[
\pi(gk)\pi(g^{-1}) = \pi(K).
\]

As a subgroup, \( \pi(K) \) must be \( K/H \) precisely because \( K \) is closed and normal in \( G \).

Thus, we have:

\[
K/H \trianglelefteq G/H,
\]

proving the backward direction.

\end{document}